%%%%%%%%%%%%%%%%%%%%%%%%%%%%%%%%%%%%%%%
% Deedy - One Page Two Column Resume
% LaTeX Template
% Version 1.2 (16/9/2014)
%
% Original author:
% Debarghya Das (http://debarghyadas.com)
%
% Original repository:
% https://github.com/deedydas/Deedy-Resume
%
% IMPORTANT: THIS TEMPLATE NEEDS TO BE COMPILED WITH XeLaTeX
%
% This template uses several fonts not included with Windows/Linux by
% default. If you get compilation errors saying a font is missing, find the line
% on which the font is used and either change it to a font included with your
% operating system or comment the line out to use the default font.
% 
%%%%%%%%%%%%%%%%%%%%%%%%%%%%%%%%%%%%%%
% 
% TODO:
% 1. Integrate biber/bibtex for article citation under publications.
% 2. Figure out a smoother way for the document to flow onto the next page.
% 3. Add styling information for a "Projects/Hacks" section.
% 4. Add location/address information
% 5. Merge OpenFont and MacFonts as a single sty with options.
% 
%%%%%%%%%%%%%%%%%%%%%%%%%%%%%%%%%%%%%%
%
% CHANGELOG:
% v1.1:
% 1. Fixed several compilation bugs with \renewcommand
% 2. Got Open-source fonts (Windows/Linux support)
% 3. Added Last Updated
% 4. Move Title styling into .sty
% 5. Commented .sty file.
%
%%%%%%%%%%%%%%%%%%%%%%%%%%%%%%%%%%%%%%%
%
% Known Issues:
% 1. Overflows onto second page if any column's contents are more than the
% vertical limit
% 2. Hacky space on the first bullet point on the second column.
%
%%%%%%%%%%%%%%%%%%%%%%%%%%%%%%%%%%%%%%


\documentclass[]{deedy-resume-openfont}
\usepackage{fancyhdr}
    
\pagestyle{fancy}
\fancyhf{}
    
\begin{document}

%%%%%%%%%%%%%%%%%%%%%%%%%%%%%%%%%%%%%%
%
%     LAST UPDATED DATE
%
%%%%%%%%%%%%%%%%%%%%%%%%%%%%%%%%%%%%%%
%\lastupdated

%%%%%%%%%%%%%%%%%%%%%%%%%%%%%%%%%%%%%%
%
%     TITLE NAME
%
%%%%%%%%%%%%%%%%%%%%%%%%%%%%%%%%%%%%%%
\namesection{Yin}{Daheng}{ \urlstyle{same}\href{mailto:yindaheng98@seu.edu.cn}{yindaheng98@seu.edu.cn} | +86 188 5189 9135 | Southeast University, Nanjing, Jiangsu}

%%%%%%%%%%%%%%%%%%%%%%%%%%%%%%%%%%%%%%
%
%     COLUMN ONE
%
%%%%%%%%%%%%%%%%%%%%%%%%%%%%%%%%%%%%%%

\begin{minipage}[t]{0.22\textwidth} 

%%%%%%%%%%%%%%%%%%%%%%%%%%%%%%%%%%%%%%
%     EDUCATION
%%%%%%%%%%%%%%%%%%%%%%%%%%%%%%%%%%%%%%

\section{Education} 
\subsection{Southeast University}
\descript{Master of Science}
\descript{Major: Network Engineering}
\location{2020.09$\sim$2023.06}
\sectionsep
\subsection{\fontsize{10pt}{10pt}\selectfont{Cambridge University}}
\descript{Overseas study tour}
\location{2017.08-2017.09}
\sectionsep
\subsection{Jiangnan University}
\descript{Bachelor of Engineering}
\descript{Major: IoT Engineering}
\location{2016.09$\sim$2020.09}
\sectionsep

\section{Transcript}
\begin{tabular*}{\linewidth}{lr@{\extracolsep{\fill}}c@{\extracolsep{\fill}}l}
    GPA(BEng)&3.59&/&4 \\
    GPA(MSc)&81.69&/&100 \\
\end{tabular*}
\sectionsep
\section{English}
\begin{tabular}{ll}
    IELTS & 6.5 \\
    CET6  & 576 \\
\end{tabular}
\sectionsep

\section{Skills}
Pytorch \\
CUDA \textbullet{} TensorRT \\
WebRTC \textbullet{} LibVPX \\
Docker \textbullet{} Kubernetes \\
Git \textbullet{} Github \textbullet{} GitLab \\
\LaTeX \textbullet{} Golang \textbullet{} Java \\

\section{Links}
Github\href{https://github.com/yindaheng98}{\bf @yindaheng98} \\
Blog:\href{http://www.yindaheng98.top}{\bf yindaheng98.top} \\
\sectionsep

\end{minipage} 
\hfill
\begin{minipage}[t]{0.76\textwidth} 

%%%%%%%%%%%%%%%%%%%%%%%%%%%%%%%%%%%%%%
%     EXPERIENCE
%%%%%%%%%%%%%%%%%%%%%%%%%%%%%%%%%%%%%%

\section{Research \& Development}
\runsubsection{Acclerate live video super-resolution with edge computing}
\descript{2021.06$\sim$currently}
\vspace{\topsep}
\begin{tightemize}
    \item Derived a parallel-friendly DNN architecture from multi-scale feature extraction structure for better multi-device acceleration in edge environment. % OK
    \item Dimensionally compressed and int8 quantized intermediate features of the DNN and encoded features into video for transmission among multiple devices. % OK
    \item Enhance the video decoder to combine low-framerate high-definition stream and high-framerate low-definition stream into high-framerate high-definition streams for smooth video playback when frame-by-frame super-resolution inference is not supported. % OK
    \item Control the inference process adaptively based on the variable batch size of DNN input and enhanced video decoder to achieve best video quality under a specific latency bound in dynamic edge environments.
    \item Related paper \textit{D. Yin et al., "WAEVSR: Enabling collaborative live video super-resolution in wide-area MEC environment,"} is submitted to WWW 2023
\end{tightemize}
\sectionsep


\section{Cooperated Research \& Development}
\runsubsection{Adaptively computational routing based on environment awareness in Compute First Network (CFN)}
\descript{2020.10$\sim$2020.12}
\begin{tightemize}
    \item Optimize the strategy of: 1) DNN layer segmentation for distributed deployment, 2) computing device selection, 3) data transmission path selection.
    \item My contribution: Development of DNN inference control testbed (\href{https://github.com/yindaheng98/DNet}{DNet}). Scheduling and synchronizing inference process among multiple computing devices.
    \item Related paper: \textit{X. Guo et al., "Exploiting the computational path diversity with in-network computing for MEC," 2022 19th Annual IEEE International Conference on Sensing, Communication, and Networking (SECON), 2022, pp. 1-9.}
\end{tightemize}
\sectionsep

\section{Projects}
\runsubsection{\href{http://yindaheng98.top:8888/ExpertField}{\bf ExpertField}}
	\descript{2019.03$\sim$currently}
    \begin{tightemize}
        \item Aiming at field data collection, both artificial(Android App) and automatic(STM32)
        \item 9 cooperators, nearly 10,000 lines of code, a complete IoT system
		\item Will be deployed in Institute of Plant Physiology \& Ecology, CAS
		\item A fast deployable combined microservice system based on Docker VM
	\end{tightemize}
	\sectionsep

	\runsubsection{\href{https://github.com/yindaheng98/GANomaly-Tensorflow}{\bf GANomaly-Tensorflow}}
	\descript{2018.10$\sim$2018.11}
	\begin{tightemize}
		\item Implement GANomaly with Tensorflow, for camera abnormality judgment
		\item Use Opencv and Tensorflow to convert video format to tfrecord dataset
	\end{tightemize}
    \sectionsep
    
	\runsubsection{\href{http://yindaheng98.top:8888/Calendars/login.html
    }{\bf Calendars}}
	\descript{2017.10$\sim$2017.12}
	\begin{tightemize}
		\item A cross-platform web application that integrates daily affairs, curriculum, achievement management, and project planning
		\item 5 cooperators, 4000 lines of code, 2 aliyun servers and 1 physical server
		\item Implemented multi-person instant process graph editing and a high-flow curriculum query component using Redis blocking queue and long polling
	\end{tightemize}
	\sectionsep

\section{Awards \& Scholarship} 
\begin{tabular}{lll}
    2018.09 & National College Mathematical Contest in Modeling & 2nd Prize(National) \\
    2017.11 & 9th National College Mathematical Contest & 2nd Prize(Provincial) \\
    2017.11 & China National Scholarship (2016-2017) & \\
    2017.05 & 14th Jiangsu College Mathematical Contest & 1st Prize\\
\end{tabular}
\sectionsep

\end{minipage} 
\end{document}  \documentclass[]{article}
