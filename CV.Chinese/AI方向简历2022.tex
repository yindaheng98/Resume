%%%%%%%%%%%%%%%%%%%%%%%%%%%%%%%%%%%%%%%
% Deedy - One Page Two Column Resume
% LaTeX Template
% Version 1.2 (16/9/2014)
%
% Original author:
% Debarghya Das (http://debarghyadas.com)
%
% Original repository:
% https://github.com/deedydas/Deedy-Resume
%
% IMPORTANT: THIS TEMPLATE NEEDS TO BE COMPILED WITH XeLaTeX
%
% This template uses several fonts not included with Windows/Linux by
% default. If you get compilation errors saying a font is missing, find the line
% on which the font is used and either change it to a font included with your
% operating system or comment the line out to use the default font.
% 
%%%%%%%%%%%%%%%%%%%%%%%%%%%%%%%%%%%%%%
% 
% TODO:
% 1. Integrate biber/bibtex for article citation under publications.
% 2. Figure out a smoother way for the document to flow onto the next page.
% 3. Add styling information for a "Projects/Hacks" section.
% 4. Add location/address information
% 5. Merge OpenFont and MacFonts as a single sty with options.
% 
%%%%%%%%%%%%%%%%%%%%%%%%%%%%%%%%%%%%%%
%
% CHANGELOG:
% v1.1:
% 1. Fixed several compilation bugs with \renewcommand
% 2. Got Open-source fonts (Windows/Linux support)
% 3. Added Last Updated
% 4. Move Title styling into .sty
% 5. Commented .sty file.
%
%%%%%%%%%%%%%%%%%%%%%%%%%%%%%%%%%%%%%%%
%
% Known Issues:
% 1. Overflows onto second page if any column's contents are more than the
% vertical limit
% 2. Hacky space on the first bullet point on the second column.
%
%%%%%%%%%%%%%%%%%%%%%%%%%%%%%%%%%%%%%%


\documentclass[]{deedy-resume-openfont}
\usepackage{fancyhdr}

\pagestyle{fancy}
\fancyhf{}
    
\begin{document}

%%%%%%%%%%%%%%%%%%%%%%%%%%%%%%%%%%%%%%
%
%     LAST UPDATED DATE
%
%%%%%%%%%%%%%%%%%%%%%%%%%%%%%%%%%%%%%%
%\lastupdated

%%%%%%%%%%%%%%%%%%%%%%%%%%%%%%%%%%%%%%
%
%     TITLE NAME
%
%%%%%%%%%%%%%%%%%%%%%%%%%%%%%%%%%%%%%%
\namesection{尹}{达恒}{ \urlstyle{same}\href{mailto:yindaheng98@seu.edu.cn}{yindaheng98@seu.edu.cn} | +86 188 5189 9135 | 东南大学,江苏南京}

%%%%%%%%%%%%%%%%%%%%%%%%%%%%%%%%%%%%%%
%
%     COLUMN ONE
%
%%%%%%%%%%%%%%%%%%%%%%%%%%%%%%%%%%%%%%

\begin{minipage}[t]{0.21\textwidth}

	%%%%%%%%%%%%%%%%%%%%%%%%%%%%%%%%%%%%%%
	%     EDUCATION
	%%%%%%%%%%%%%%%%%%%%%%%%%%%%%%%%%%%%%%

	\section{教育经历}
	\subsection{东南大学}
	\descript{学术学位硕士}
	\descript{计算机科学与技术}
	\location{2020.09-2023.06}
    \sectionsep
    \subsection{\fontsize{10pt}{10pt}\selectfont{Cambridge University}}
	\descript{暑期游学}
	\descript{Global Innovation and Leaderdship}
	\location{2017.08-2017.09}
    \sectionsep
	\subsection{江南大学}
	\descript{工学学士学位}
	\descript{物联网工程}
	\location{2016.09-2020.06}
    \sectionsep

	%%%%%%%%%%%%%%%%%%%%%%%%%%%%%%%%%%%%%%
	%     COURSEWORK
	%%%%%%%%%%%%%%%%%%%%%%%%%%%%%%%%%%%%%%


	%%%%%%%%%%%%%%%%%%%%%%%%%%%%%%%%%%%%%%
	%     SKILLS
	%%%%%%%%%%%%%%%%%%%%%%%%%%%%%%%%%%%%%%

	\section{成绩}
	\subsection{绩点}
	\begin{tabular*}{\linewidth}{lr@{\extracolsep{\fill}}c@{\extracolsep{\fill}}l}
		本科绩点&3.59&/&4 \\
		硕士均分&81.69&/&100 \\
	\end{tabular*}
	\sectionsep
	\subsection{英语}
	\begin{tabular}{ll}
		IELTS & 6.5 \\
		CET6  & 576 \\
	\end{tabular}
	\sectionsep
    
	%%%%%%%%%%%%%%%%%%%%%%%%%%%%%%%%%%%%%%
	%     LINKS
	%%%%%%%%%%%%%%%%%%%%%%%%%%%%%%%%%%%%%%

	\section{链接}
	Github\href{https://github.com/yindaheng98}{\bf @yindaheng98} \\
	Blog:\href{http://www.yindaheng98.top}{\bf yindaheng98.top} \\
	\sectionsep

	\section{技能}
	Pytorch \\
	CUDA \textbullet{} TensorRT \\
	WebRTC \textbullet{} LibVPX \\
	Docker \textbullet{} Kubernetes \\
	Git \textbullet{} Github \textbullet{} GitLab \\
	\LaTeX \textbullet{} Golang \textbullet{} Java \\
	\sectionsep

	%%%%%%%%%%%%%%%%%%%%%%%%%%%%%%%%%%%%%%
	%
	%     COLUMN TWO
	%
	%%%%%%%%%%%%%%%%%%%%%%%%%%%%%%%%%%%%%%

\end{minipage}
\hfill
\begin{minipage}[t]{0.77\textwidth}

	%%%%%%%%%%%%%%%%%%%%%%%%%%%%%%%%%%%%%%
	%     EXPERIENCE
	%%%%%%%%%%%%%%%%%%%%%%%%%%%%%%%%%%%%%%
	% TODO 放老师看得懂的东西!

	\section{主要科研项目}
	\runsubsection{\bf 边缘计算环境下实时视频流超分辨率推断加速}
	\descript{2021.06至今}
	\vspace{\topsep}
	\begin{tightemize}
		\item 改进常见的多尺度U-net结构实现适合多机并行加速的分体式视频超分辨率神经网络
		\item 对神经网络模型的中间特征进行维度压缩和int8量化后编码为视频以便在多机间传输
		\item 改进视频解码器实现高清帧和低清流合成高清流的功能以在无法逐帧推断时保证视频流畅
		\item 基于模型输入帧数可变的性质自适应地控制推断过程以在动态的边缘环境中保障用户体验
		\item 相关论文:D. Yin et al., "WAEVSR: Enabling collaborative live video super-resolution in wide-area MEC environment," 已投稿至WWW2023
	\end{tightemize}
    \sectionsep

	\section{合作科研项目}
	%\runsubsection{边缘计算环境下的多目标跟踪模型推理加速}
	%\descript{2021.03$\sim$2021.05}
	%\begin{tightemize}
	%	\item 在边缘计算环境下优化多目标跟踪任务的模型选择和图像区域划分策略
	%	\item M. Liu et al., "Accelerating Multi-Object Tracking in Edge Computing Environment with Time-Spatial Optimization," 2021 Ninth International Conference on Advanced Cloud and Big Data (CBD), 2022, pp. 279-284.
	%	\item A. Tang et al., "Accelerate High-resolution Multi-Object Tracking on Edge Devices with Region-aware Parallel Inference," 已投稿至WWW2023
	%	\item 我的贡献:多目标跟踪模型的量化加速
	%\end{tightemize}
    %\sectionsep
	
	\runsubsection{\bf 基于算力网络环境感知的自适应计算路由}
	\descript{2020.10$\sim$2020.12}
	\begin{tightemize}
		\item 在算力网络环境下优化神经网络模型的切分以及计算设备和数据传输路径的选择策略
		\item 合作贡献:推断控制实验系统的开发与维护 (\href{https://github.com/yindaheng98/DNet}{github.com/yindaheng98/DNet})
		\item 相关论文:X. Guo et al., "Exploiting the computational path diversity with in-network computing for MEC," 2022 19th Annual IEEE International Conference on Sensing, Communication, and Networking (SECON), 2022, pp. 1-9.
	\end{tightemize}
    \sectionsep


    \section{主要实践项目}
	\runsubsection{\bf 竞赛 TensorRT Hackathon 2022 优胜奖}
	\descript{2022.03$\sim$2022.5}
	\location{NVIDIA | 阿里云天池}
	\begin{tightemize}
		\item 基于TensorRT对超分辨率神经网络模型ELAN进行float16和int8量化加速
		\item 在float16量化后的网络结构中搜索并排除对精度影响较大的层,使得误差下降了75\%
		\item 对模型进行int8 QAT量化,在保证输出图像质量不变的前提下达到50\%的加速比
		\item 项目地址:\href{https://github.com/liu-mengyang/trt-elan}{github.com/liu-mengyang/trt-elan}
	\end{tightemize}
	\sectionsep

	\runsubsection{\bf 横向课题 动态网络下边缘智能云边端协同任务调度技术研究}
	\descript{2021.09至今}
	\location{中国移动有限公司 | 东南大学}
	\begin{tightemize}
		\item 负责整体技术选型、系统架构和协作机制设计以及系统运维,并主导网络仿真模块开发
		\item 实践了在开源项目中学习到的大型软件设计思路和协作开发机制
		\item 设计并实现了基于Kubernetes和Nginx的微服务架构并以此整合团队成员的开发工作
		\item 设计并主导运作了基于Gitlab的协作开发流程和基于Docker的联合调试与交付流程
	\end{tightemize}
	\sectionsep

	\runsubsection{\bf 竞赛 TensorRT Hackathon 2021 三等奖}
	\descript{2021.03$\sim$2021.5}
	\location{NVIDIA | 阿里云天池}
	\begin{tightemize}
		\item 基于TensorRT对多目标跟踪算法FairMOT进行量化加速
		\item 编写TensorRT插件实现16位浮点型DCNv2算子,单视频输入计算速度提升2.36倍
		\item 将Pytorch模型参数通过API导入TensorRT中并完成对齐
		\item 项目地址:\href{https://github.com/liu-mengyang/trt-fairmot}{github.com/liu-mengyang/trt-fairmot}
	\end{tightemize}
	\sectionsep

	\runsubsection{\bf 横向课题 面向工业互联网的智能云端协作关键技术及系统}
	\descript{2021.04$\sim$2021.09}
	\location{国家重点研发计划 | 南京南钢钢铁联合有限公司 | 东南大学}
	\begin{tightemize}
		\item 负责轧钢厂产线数据监视系统的开发工作
		\item 基于Unity和Node.js实现了轧钢厂生产线数据在三维产线模型中的实时展示
	\end{tightemize}
	\sectionsep

	%\runsubsection{机器学习算法库mlpack在神威·太湖之光系统上的移植}
	%\descript{}
	%\location{2019.7$\sim$2019.12 | 江南大学超算俱乐部/神威·太湖之光软件部门}
	%\vspace{\topsep}
	%\begin{tightemize}
	%	\item 参与kernels部分的移植以及超算俱乐部成员培训
	%	\item 依托此项目独立完成的太湖之光平台开发教程是目前俱乐部的内部培训教材之一
	%\end{tightemize}
    %\sectionsep

	\section{本科期间的竞赛和奖学金}
    \begin{tabular}{lll}
        %2019.05 & 全国大学生服务外包创新创业大赛 & 三等奖 \\
        %2019.04 & 美国大学生数学建模竞赛 & S奖 \\
        2018.09 & 全国大学生数学建模竞赛 & 国家级二等奖 \\
        2017.11 & 第九届全国大学生数学竞赛(非数学类) & 江苏赛区二等奖 \\
        2017.05 & 江苏省普通高等学校第十四届数学竞赛 & 本科一级组一等奖\\
        2017.11 & 2016-2017年度国家奖学金 & \\
	\end{tabular}
    \sectionsep
\end{minipage}
\end{document}
