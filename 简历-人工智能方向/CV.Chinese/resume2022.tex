%%%%%%%%%%%%%%%%%%%%%%%%%%%%%%%%%%%%%%%
% Deedy - One Page Two Column Resume
% LaTeX Template
% Version 1.2 (16/9/2014)
%
% Original author:
% Debarghya Das (http://debarghyadas.com)
%
% Original repository:
% https://github.com/deedydas/Deedy-Resume
%
% IMPORTANT: THIS TEMPLATE NEEDS TO BE COMPILED WITH XeLaTeX
%
% This template uses several fonts not included with Windows/Linux by
% default. If you get compilation errors saying a font is missing, find the line
% on which the font is used and either change it to a font included with your
% operating system or comment the line out to use the default font.
% 
%%%%%%%%%%%%%%%%%%%%%%%%%%%%%%%%%%%%%%
% 
% TODO:
% 1. Integrate biber/bibtex for article citation under publications.
% 2. Figure out a smoother way for the document to flow onto the next page.
% 3. Add styling information for a "Projects/Hacks" section.
% 4. Add location/address information
% 5. Merge OpenFont and MacFonts as a single sty with options.
% 
%%%%%%%%%%%%%%%%%%%%%%%%%%%%%%%%%%%%%%
%
% CHANGELOG:
% v1.1:
% 1. Fixed several compilation bugs with \renewcommand
% 2. Got Open-source fonts (Windows/Linux support)
% 3. Added Last Updated
% 4. Move Title styling into .sty
% 5. Commented .sty file.
%
%%%%%%%%%%%%%%%%%%%%%%%%%%%%%%%%%%%%%%%
%
% Known Issues:
% 1. Overflows onto second page if any column's contents are more than the
% vertical limit
% 2. Hacky space on the first bullet point on the second column.
%
%%%%%%%%%%%%%%%%%%%%%%%%%%%%%%%%%%%%%%


\documentclass[]{deedy-resume-openfont}
\usepackage{fancyhdr}

\pagestyle{fancy}
\fancyhf{}
    
\begin{document}

%%%%%%%%%%%%%%%%%%%%%%%%%%%%%%%%%%%%%%
%
%     LAST UPDATED DATE
%
%%%%%%%%%%%%%%%%%%%%%%%%%%%%%%%%%%%%%%
%\lastupdated

%%%%%%%%%%%%%%%%%%%%%%%%%%%%%%%%%%%%%%
%
%     TITLE NAME
%
%%%%%%%%%%%%%%%%%%%%%%%%%%%%%%%%%%%%%%
\namesection{尹}{达恒}{ \urlstyle{same}\href{mailto:yindaheng98@seu.edu.cn}{yindaheng98@seu.edu.cn} | +86 188 5189 9135 | 东南大学,江苏南京}

%%%%%%%%%%%%%%%%%%%%%%%%%%%%%%%%%%%%%%
%
%     COLUMN ONE
%
%%%%%%%%%%%%%%%%%%%%%%%%%%%%%%%%%%%%%%

\begin{minipage}[t]{0.25\textwidth}

	%%%%%%%%%%%%%%%%%%%%%%%%%%%%%%%%%%%%%%
	%     EDUCATION
	%%%%%%%%%%%%%%%%%%%%%%%%%%%%%%%%%%%%%%

	\section{教育经历}
	\subsection{东南大学}
	\descript{学术学位硕士}
	\descript{计算机科学与技术}
	\location{2020.09-2023.06}
    \sectionsep
	\subsection{江南大学}
	\descript{工学学士学位}
	\descript{物联网工程}
	\location{2016.09-2020.06}
    \sectionsep

	%%%%%%%%%%%%%%%%%%%%%%%%%%%%%%%%%%%%%%
	%     COURSEWORK
	%%%%%%%%%%%%%%%%%%%%%%%%%%%%%%%%%%%%%%


	%%%%%%%%%%%%%%%%%%%%%%%%%%%%%%%%%%%%%%
	%     SKILLS
	%%%%%%%%%%%%%%%%%%%%%%%%%%%%%%%%%%%%%%

	\section{成绩}
	\subsection{绩点}
	\begin{tabular}{llll}
		本科绩点&3.59&/&4 \\
		硕士均分&81.69&/&100 \\
	\end{tabular}
	\sectionsep
	\subsection{英语}
	\begin{tabular}{ll}
		IELTS & 6.5 \\
		CET6  & 576 \\
	\end{tabular}
	\sectionsep
    
	%%%%%%%%%%%%%%%%%%%%%%%%%%%%%%%%%%%%%%
	%     LINKS
	%%%%%%%%%%%%%%%%%%%%%%%%%%%%%%%%%%%%%%

	\section{链接}
	\begin{tabular}{ll}
		Github & \href{https://github.com/yindaheng98}{\bf @yindaheng98} \\
		Blog  & \href{http://www.yindaheng98.top}{\bf yindaheng98.top} \\
	\end{tabular}
	\sectionsep

	%%%%%%%%%%%%%%%%%%%%%%%%%%%%%%%%%%%%%%
	%
	%     COLUMN TWO
	%
	%%%%%%%%%%%%%%%%%%%%%%%%%%%%%%%%%%%%%%

\end{minipage}
\hfill
\begin{minipage}[t]{0.73\textwidth}

	%%%%%%%%%%%%%%%%%%%%%%%%%%%%%%%%%%%%%%
	%     EXPERIENCE
	%%%%%%%%%%%%%%%%%%%%%%%%%%%%%%%%%%%%%%

	\section{主要科研项目}
	\runsubsection{边缘计算场景下实时视频流超分辨率多机协同推理优化}
	\descript{2021至今}
	\vspace{\topsep}
	\begin{tightemize}
		\item 参与kernels部分的移植以及超算俱乐部成员培训
		\item 依托此项目独立完成的太湖之光平台开发教程是目前俱乐部的内部培训教材之一
	\end{tightemize}
    \sectionsep

    \section{主要实践项目}
	\runsubsection{\href{https://github.com/liu-mengyang/trt-elan}{\bf TensorRT Hackathon 2022 竞赛 优胜奖}}
	\descript{2022.03$\sim$2022.5}
	\begin{tightemize}
		\item 基于TensorRT对图像超分辨率算法ELAN进行量化加速
		\item 项目地址:\href{https://github.com/liu-mengyang/trt-elan}{github.com/liu-mengyang/trt-elan}
	\end{tightemize}
	\sectionsep

	\runsubsection{\href{https://github.com/liu-mengyang/trt-fairmot}{\bf TensorRT Hackathon 2021 竞赛 三等奖}}
	\descript{2021.03$\sim$2021.5}
	\begin{tightemize}
		\item 基于TensorRT对视频跟踪算法FairMOT进行量化加速
		\item 项目地址:\href{https://github.com/liu-mengyang/trt-fairmot}{github.com/liu-mengyang/trt-fairmot}
	\end{tightemize}
	\sectionsep

	%\runsubsection{机器学习算法库mlpack在神威·太湖之光系统上的移植}
	%\descript{}
	%\location{2019.7$\sim$2019.12 | 江南大学超算俱乐部/神威·太湖之光软件部门}
	%\vspace{\topsep}
	%\begin{tightemize}
	%	\item 参与kernels部分的移植以及超算俱乐部成员培训
	%	\item 依托此项目独立完成的太湖之光平台开发教程是目前俱乐部的内部培训教材之一
	%\end{tightemize}
    %\sectionsep

	\section{其他实践项目}
    \begin{tabular}{lll}
        %2019.05 & 全国大学生服务外包创新创业大赛 & 三等奖 \\
        2019.04 & 美国大学生数学建模竞赛 & S奖 \\
        2018.09 & 全国大学生数学建模竞赛 & 国家级二等奖 \\
        2017.11 & 第九届全国大学生数学竞赛(非数学类) & 江苏赛区二等奖 \\
        2017.05 & 江苏赛普通高等学校第十四届数学竞赛 & 本科一级组一等奖\\
	\end{tabular}
    \sectionsep

	\section{奖学金}
    \begin{tabular}{lll}
        2018.11 & 江南大学2016-2017学年学业奖学金 & 一等 \\
        2017.11 & 2016-2017年度国家奖学金 & \\
	\end{tabular}
    \sectionsep
    
    \section{个人陈述}
    \subsection{研究兴趣}
    \vspace{\topsep}
	\begin{tightemize}
		\item 理论研究兴趣:机器学习算法、神经网络可解释性、类人脑计算模型
		\item 应用研究兴趣:医疗数据挖掘、物联网云数据融合、大规模并行计算
	\end{tightemize}
    \subsection{个人特质}
    \vspace{\topsep}
    \begin{tightemize}
        \item 历经两个数学竞赛和两个数学建模竞赛,具有较良好的数学抽象和建模能力
		\item 大学4年间总代码量近30000行(不计Latex和HTML),具有良好的编码和架构能力
		\item 领导完成多人开发项目4个,合作开发项目2个,具有良好的合作和领导能力
	\end{tightemize}
\end{minipage}
\end{document}
